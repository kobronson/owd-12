
\documentclass[a4paper, 10pt]{article}


\usepackage[polish]{babel}
\usepackage[utf8]{inputenc}
\usepackage[OT4]{fontenc}
\usepackage{geometry}
\usepackage{ulem}
\usepackage{amsfonts}
\usepackage{lscape}	

\usepackage[fleqn]{amsmath}



\usepackage{mathtools}
\RequirePackage{url}


\setlength{\parindent}{0cm}
\setlength{\parskip}{3mm plus1mm minus1mm}

\geometry{verbose,a4paper,tmargin=2.1cm,bmargin=2.1cm,lmargin=2.1cm,rmargin=2.1cm}
\usepackage{graphicx} % wstawianie obrazkow

\usepackage{colortbl}






%%%%%%%%%%%%%%%%%%%%%%%%%%%%%%%%%%%%%%%%%%%%%%%%


\title{{\bf {Optymalizacja we wspomaganiu decyzji}} \\ {\large Sprawozdanie}}
\date{\today}
\author{Filip Nabrdalik}

%%%%%%%%%%%%%%%%%%%%%%%%%%%%%%%%%%%%%%%%%%%%%%%%


\begin{document}
\bibliographystyle{plain}


%%%%%%%

\maketitle 


%%%%%%%%%%%%%%%%%%%%%%%%%%%%%


\newcommand{\ang}[1]{(ang. {\em #1}\/)}
\newcommand{\e}[1]{{\em #1}\/}





\section{Skrócona treść zadania}

{\bf OD Projekt \hfill Dane numer 12\\}
Zakłady chemiczne produkują 3 rodzaje produktów P1, P2, P3. Wykorzystuje się do tego dwa rodzaje surowca S1 i S2 dostępnego w ilościach,
odpowiednio 11000 i 12000 ton dziennie. Ceny surowca S1 i S2 wynoszą, odpowiednio 130 i 110 zł za tonę.

Surowce są poddawane wstępnej obróbce w przygotowalni o całkowitej dziennej przepustowości 16700 ton. 
w wyniku tego powstają trzy półprodukty: D1,D2 i D3. Ilość poszczególnych półproduktów w zależności 
od surowca kształtuje się następująco 

\begin{center}
    \begin{tabular}{ | c | c | c | c |}
    \hline
    Surowiec & D1 & D2 & D3 \\ \hline
	S1 & 0,4 & 0,3 & 0,3 \\ \hline
	S2 & 0,1 & 0,8 & 0,1 \\ \hline
    \end{tabular}
\end{center}


D1 może być bezpośrednio użyty do produkcji P1 i P3. Natomiast D2 i D3 mogą być wykorzystane
do bezpośredniej produkcji P2 lub przeznaczone do dalszej obróbki w zakładzie uwodornienia. Koszt przygotowania surowca zależny jest od wielkości 
dziennego przerobu danego surowca i jest przedstawiony w tabeli poniżej


\begin{center}
    \begin{tabular}{ | l | l | l |}
    \hline
    & Dzienny przerób & Koszt przygotowania \\ \hline
	S1 & od 2041 ton włącznie & 19zł/tonę \\ \hline
	& od 2041 do 6439 włącznie & 2041*19zł+14zł za kazdą tonę powyżej progu 2014 ton \\ \hline
	& od 6439 & 2041*19zł+(6439-2041)*14zł+10zł za kazdą tonę powyżej progu 6439 ton \\ \hline
	S2 & od 2733 ton włącznie & 12zł/tonę \\ \hline
	& od 2733 do 6751 włącznie & 2733*12zł+16zł za kazdą tonę powyżej progu 2014 ton \\ \hline
	& od 6751 & 2733*12zł+(6751-2733)*16zł+20zł za kazdą tonę powyżej progu 6751 ton \\ \hline
    \end{tabular}
\end{center}



Zakład uwodornienia o przepustowości dziennej 8637 ton wytwarza półprodukty K1 i K2. Ilość poszczególnych półproduktów w zależności od surowca kształtuje się następująco



\begin{center}
    \begin{tabular}{ | c | c | c |}
    \hline
    Półprodukt & D2 & D3 \\ \hline
	K1 & 0,6 & 0,6 \\ \hline
	K2 & 0,4 & 0,4 \\ \hline
    \end{tabular}
\end{center}


Półprodukty Ki i K2 mogą być użyte do produkcji P1 lub P2, ale nie mogą być użyte do wytworzenia P3. Jeśli 
kład uwodornienia pracuje, dzienny koszt jego pracy (niezależnie od ilości przetworzonych produktów) wynosi
15 tys. zł. Jeśli nie pracuje, dzienny koszt jest zerowy. Cena sprzedaży hurtowej wynosi: 193 zł/tonę P1, 136 zł/tonę i
100 zł/tonę P3. Zawarte umowy wymagają dostarczenia co najmniej 4917 ton każdego produktu.

Dany produkt finalny jest wytwarzany z półproduktów D1, D2, D3, K1 lub K2 bezpośrednio bez utraty masy, jednak zgodnie
z zasadami opisanymi wcześniej.

Należy zaprojektować system wspomagania decyzji, w którym są 4 kryteria: całkowity koszt produkcji oraz względne niedobory
każdego produktu. System powinien wykorzystywać metodę punktu odniesienia do agregacji kryteriów.



\section{Model matematyczny}



{\bf Zmienne decycyjne\\}
\hfill \\
Ilość surowca $i$ sprzedawanego po koszcie $j$
$$s_{i,j} \ge 0, \; s_{i,j} \in \mathbb Z_{\ge 0}$$ 
Ilość półproduktu D użytego do produkcji P
$$d_{i,j}  \ge 0,\; d_{i,j} \in \mathbb Z_{\ge 0}$$  
Ilość półproduktu K użytego do produkcji P i K
$$k_{i,j}  \ge ,\; k_{i,j} \in \mathbb Z_{\ge 0}$$  



{\bf Produkty\\}

\hfill \\ 
Ilość wytworzonych produktów
$$p_{k}  \ge 0$$ 

Półprodukt D1 i suma półproduktów K przeznaczonych do produkcji P1  wytworzona ilość produktu P1 = D1 + (K11 + K12) 
$$p_1 = d_{1,1} + \sum_{i=1}^{K} k_{i,1} $$  

wytworzona ilość produktu P2 = ((D2 + D3)-(K1+K2)) + (K12 + K22) , pierwszy nawias reprezentuje półprodukty D, 
a drugi K przeznaczone do produkcji P2

$$p_2 =  (\sum_{i=0}^{D} (d_{2,i} + d_{3,i}) -  \sum_{j=1}^{P}  \sum_{i=1}^{K} k_{i,j}) +   \sum_{i=1}^{P} k_{i,2}  $$ 
wytworzona ilość produktu P3 = D3
$$p_3 = d_{1,3} $$ 

{\bf Surowce i półprodukty\\}
\hfill \\ 

Dzienny limit surowców
$$\forall_{i \in S},\; \sum_{j=1}^{3}s_{i,j} \le  LS_i$$  

Przepustowość przygotowalni D, D1+D2+D3 
$$\sum_{i=1}^{D}\sum_{j=1}^{P}  d_{i,j} \le  LD$$

Zapotrzebowanie przygotowalni  

$$\forall_{i \in D},\; \sum_{j=1}^{P} d_{i,j} = \sum_{k=1}^{S} (ZP_{i,k} \cdot \sum_{l=1}^{3}s_{k,l}) $$ 

Przepustowość uwodornienia K1+K2

$$\sum_{j=1}^{P}  \sum_{i=1}^{K} k_{i,j} \le  LU$$

Zapotrzebowanie uwodornienia 


$$\forall_{i \in K},\; \sum_{j=1}^{P} k_{i,j} =  \sum_{j=2}^{P}  (d_{i,j} \cdot ZU_{i,j})$$ 

{\bf Koszty\\}
\hfill \\ 
Koszty obróbki S1 (progi - koszty malejące):

$$ PS1_{1}\cdot b_{1} \le s_{1,1} \le  PS1_{1}  $$
$$ PS1_{2}\cdot b_{2} \le s_{1,2} \le  PS1_{1}\cdot b_{1}  $$
$$ PS1_{3}\cdot b_{2} \le s_{1,3}   $$
$$ \forall_{i={1,2}}  \;  0 \le b_{i} \le 1,\;b_{i} \in \mathbb Z_{\ge 0},   $$

Koszty obróbki S2 (progi - koszty rosnące):

$$ \forall_{i={1,2}} \;  \sum_{j =1}^{P} s_{i,j} \le PS2_{i}$$

Koszty uwodornienia (flaga dodatniości):

\begin{flalign*}
  bu \le 	\sum_{j=1}^{P}  \sum_{i=1}^{K} k_{i,j}  \le LU \cdot bu , \; 0 \le bu \le 1,\;bu \in \mathbb Z_{\ge 0},
\end{flalign*}

{\bf Kryteria\\}

Koszt całkowity:
\begin{flalign*}
 c_1 = (\sum_{i=1}^{S}( {KS}_i \cdot \sum_{j=1}^{3}s_{i,j}))  + (\sum_{i=1}^{S} \sum_{j=1}^{3}(s_{i,j}\cdot \mathit{KO}_{i,j})) + bu\cdot \mathit{KU}
\end{flalign*}
Względne niedobory produktów P1, P2, P3:
\begin{flalign*}
 c_2 = \frac{(Z - p_1)}{Z}  \\
 c_3 = \frac{(Z - p_2)}{Z}  \\
 c_4 = \frac{(Z - p_3)}{Z}  \\
\end{flalign*}




\subsection{Metoda ważenia ocen}

W początkowym etapie prac metoda ważenie ocen służyła do weryfikacji modelu. Kryteria niedoborów poszczególnych produktów zostały zagregowane w jedno.
Metoda wagowa załączona do sprawozdania dla porównania z metodą punktu referencyjnego.

$$ \forall_{i={1,2},\; w_i \ge 0} , \; min: w_1 \cdot c_1+w_2 \cdot (c_2+c_3+c_4) $$




\subsection{Klasyczna metoda punktu odniesienia}

Parametry:
\begin{flalign*}
\beta = 0.001\\
\mathcal{E} = 0.0005\\
\forall_{i \in C} \; \lambda_i =[1, 1, 1, 1]\\
\end{flalign*}

Prarametr aspiracji $\forall_{c \in C} \;  a_c$. \\
Wartość funkcji osiągnięcia dla każdego z kryteriów $\forall_{c \in C} \;  z_c$. Minimalna funkcja osiągnięcia $v$. 



\begin{flalign*}
v \le z_c \;\forall_{c \in C}\\
 z_c \le \cdot \beta \cdot \lambda_c \cdot (c_c - a_c) \;\forall_{c \in C} \;\\
z_c \le \cdot \lambda_c \cdot (c_c - a_c)\; \forall_{c \in C} \;\\
max: \; v+ \mathcal{E} \sum_{c \in C} z_c  
\end{flalign*}

\section{Wyniki}

Dla symulacji przeprowadzonej metodą ważenia ocen nie uzyskano zadowalających wyników. W 50 krokach
kryterium kosztu zdecydowanie przeważa nad niedoborami produktów w związku z czym produkcja się nie odbywa. Dopiero dla zerowej wagi
kosztu otrzymuje utopię dla zagregowanych niedoborów produktów.  

\begin{center}
    \begin{tabular}{ | l | l | l |}
    \hline
    Kryterium & Utopia & Nadir \\ \hline
	$c_1$ (koszt) & 0 & 1844460 \\ \hline
	$c_2$ (wzg. niedobór P1) &-1.63372 &  1\\ \hline
	$c_3$ (wzg. niedobór P2) &-0.699207 & 1\\ \hline
	$c_4$ (wzg. niedobór P3) & 0.0654871& 1	\\ \hline
    \end{tabular}
\end{center}





\newpage

\begin{landscape}
\subsection{Ważenie ocen}


\begin{center}
    \begin{tabular}{ | c | c | c | c | c | c | c | c | c | c |}
    \hline
    $w_k$ & $w_n$ & $S$ & $D$ & $K$ & $P$ & Koszt & Wzg. Niedobór P1  & Wzg. Niedobór P2 & Wzg. Niedobór P3 \\ \hline
	0 & 1 &  [11000, 1950] &  [4595, 4860, 3495] &  [5013, 3342] &  [3342, 5013, 4595] & 1844460 & 0.320317 & -0.0195241 & 0.0654871\\ \hline
	$w_k \ne 0$ & $w_n \ne 1$ &  [0, 0] &  [0, 0, 0] &  [0, 0] &  [0, 0, 0] & 0 & 1 & 1 & 1 \\ \hline
    \end{tabular}
\end{center}






\subsection{Punkt referencyjny}



\begin{center}
    \begin{tabular}{ | l | c | c | c | c | c | c | c | c |c |}
    \hline
	
    $a$ & $S$ & $D$ & $K$ & $P$ & Koszt & Wzg. nd. P1  & Wzg. nd. P2 & Wzg. nd. P3 & $zysk-koszt$ \\ \hline	 
	 [1844460, 1, 1, 1] &  [11000, 1950] & [4595, 4860, 3495] & [5013, 3342] & [4310, 4320, 4320] & 1844461 & 0.1234  & 0.121  & 0.121  & 6889 \\  \hline
	 [0, -1.633, -0.699, 0.065] &  [10990, 1940] & [4590, 4849, 3491] & [5004, 3336] & [8657, 4043, 230] & 1841761 & -0.760  & 0.177  & 0.953  & 401888 \\  \hline
	 [1800000, -1.633, -0.699, 0.065] &  [10760, 2110] & [4515, 4916, 3439] & [5013, 3342] & [8664, 4004, 202] & 1831077 & -0.762  & 0.1857  & 0.958  & 405819 \\  \hline
	 [1800000, -2, -0.699, 0.065] &  [10750, 2100] & [4510, 4905, 3435] & [5004, 3336] & [9107, 3743, 0] & 1828273 & -0.852  & 0.238  & 1  & 438426 \\  \hline
	 [1800000, -3, -0.699, 0.065] &  [11000, 1950] & [4595, 4860, 3495] & [5013, 3342] & [12950, 0, 0] & 1844461 & -1.633  & 1  & 1  & 654889 \\  \hline
	 [1500000, -1.633, -0.699, 0.065] &  [10970, 1970] & [4585, 4867, 3488] & [5013, 3342] & [8626, 4042, 272] & 184284 & -0.754  & 0.1780  & 0.944  & 398881 \\  \hline
	 [1500000, -1, -1, -1] &  [11000, 1950] & [4595, 4860, 3495] & [5013, 3342] & [4310, 4320, 4320] & 1844461 & 0.123  & 0.121  & 0.121  & 6889\\ \hline
	 [1800000, 0, 0, 0] &  [11000, 1950] & [4595, 4860, 3495] & [5013, 3342] & [4310, 4320, 4320] & 1844461 & 0.123  & 0.121  & 0.121  & 6889 \\ \hline
	 [1800000, 0, 0.5, 0.5] &  [10970, 1970] & [4585, 4867, 3488] & [5013, 3342] & [6020, 3494, 3426] & 1842837 & -0.224  & 0.289  & 0.303  & 136807 \\ \hline
	 [1000000, -1, 0.5, 0.5] &  [10940, 1990] & [4575, 4874, 3481] & [5013, 3342] & [9232, 1857, 1841] & 1841261 & -0.877  & 0.622  & 0.625  & 377167 \\ \hline
	 [1000000, 0.5, -1, 0.5] &  [10880, 2030] & [4555, 4888, 3467] & [5013, 3342] & [2258, 8354, 2298] & 1837809 & 0.540  & -0.699  & 0.532  & -36071 \\ \hline
	 [1000000, 0.5, 0.5, -1] &  [10970, 1970] & [4585, 4867, 3488] & [5013, 3342] & [4184, 4172, 4584] & 1842861 & 0.149  & 0.151  & 0.067  & -9557\\ \hline
	 [1000000, -1, -1, 1] &  [11000, 1950] & [4595, 4860, 3495] & [5013, 3342] & [4598, 8352, 0] & 1844461 & 0.064  & -0.6986  & 1  & 178825 \\ \hline
	 [1000000, -2, 0, -1] &  [10850, 2050] & [4545, 4895, 3460] & [5013, 3342] & [9771, 1, 3128] & 1836129 & -0.987  & 0.9998  & 0.363 & 362610 \\ \hline
	 
	 
    \end{tabular}
\end{center}




\end{landscape}
\section{Załączniki i instrukcja}






\end{document}


